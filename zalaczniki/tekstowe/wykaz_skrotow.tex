\nonumsection{Wykaz ważniejszych skrótów i~oznaczeń}
\suppressfloats[t]

\begin{description}[\setleftmargin{65pt}\setlabelstyle{\bfseries}]
    \leftskip=1cm
    \item[ADS-B] System automatycznego nadzoru rozgłoszeniowy (ang. Automatic dependent surveillance – broadcast) - system podający położenie „własnego” statku powietrznego innym statkom powietrznym oraz kontroli ruchu lotniczego (ADS-B out), a także odbierający sygnały od innych uczestników ruchu lotniczego.
    \item[EASA] Europejska Agencja Bezpieczeństwa Lotniczego (ang. European Aviation Safety Agency) - agencja zajmująca się problemami bezpieczeństwa lotniczego w Europie.
    \item[ICAO] Organizacja Międzynarodowego Lotnictwa Cywilnego (ang. International Civil Aviation Organization) - agencja odpowiedzialna za wdrażanie międzynarodowych przepisów regulujących bezpieczeństwo w ruchu powietrznym.
    \item[IoE] Internet wszechrzeczy (ang. Internet of Everything) - koncepcja która rozszerza internet rzeczy (IoT) i kładzie nacisk na komunikację maszyna-maszyna, opisując bardziej skomplikowany system który obejmuje również ludzi i procesy.
    \item[IoT] Internet rzeczy (ang. Internet of Things) - koncepcja, wedle której przedmioty mogą gromadzić, przetwarzać, oraz wymieniać dane za pośrednictwem szeroko rozumianej sieci.
    \item[PAŻP] Polska Agencja Żeglugi Powietrznej (ang. Polish Air Navigation Services Agency PANSA) - agencja zarządza przestrzenią powietrzną, przepływem ruchu lotniczego i zapewnieniem służb ruchu lotniczego na terenie Polski.
    \item[RPA] Samolot pilotowany zdalnie (ang. Remotely-piloted aircraft) 
    \item[RSSI] Wskaźnik siły sygnału w urządzeniu odbiorczym (ang. Received Signal Strehgth Indication) \cite{rssi}.	    
    \item[UAS] Bezzałogowy systemy statków powietrznych (ang. Unmanned Aircraft Systems) - szeroko rozumiane systemy obsługujące statki powietrzne, działające na pokładzie statków powietrznyh w całości lub części.
    \item[UAV] Bezzałogowy statek powietrzny (ang. Unmanned Aerial Vehicle) - dron, który nie wymaga do lotu załogi obecnej na pokładzie oraz nie ma możliwości zabierania pasażerów.
    \item[UAVO] Operator bezzałogowego stateku powietrznego (ang. Unmanned Aerial Vehicle Operator) - 
w polskim prawie oznacza świadectwo kwalifikacji operatorów dronów, uprawniające do ich wykorzystywania w celach  innych niż loty sportowe i rekreacyjne.
    \item[VLOS] (ang. Visual Line of Sight) - licencja uprawniająca do lotów w zasięgu wzroku poz warunkiem zachowania bezpiecznej odległości i odpowiedniego oznakowania. Dron musi mieć tabliczkę identyfikacyjną, a operator kamizelkę.
\end{description}
