\newpage
\begin{center}
 {\large\bf Abstract} \\
\vskip 1cm
Topic:\\
\textit{\bf Use of internet of things and unmanned aerial vehicles as assistance in the rescue operations}
\end{center}
\vskip 1cm

This paper describes the process of creating a system, using the internet of things, and unmanned aerial vehicles in support of the rescue operations. I start with theoretical introduction, a description of the technologies, and the way I combine both technologies, then I dive into my implementation, description of used solutions, problems, at the end I present the results of whole system tests and conclusions that I have learned from this project.

The topic is difficult, I need to combine two, fairly new technologies, unmanned aerial vehicles and the internet of things, to create one system, which is intended, to help people. I encountered a lot problems on my way, starting from security, autonomous flight, UAV's short flight time etc. I tried to describe, how I dealt with them.

In the process of creating the system, several programs were created, the program for Android operating system, program for Pixhawk flight controller was modified, and ground station program in the Django framework was done from scratch. I do not include the source code here, but everything is available online in public Github repositories \url{https://github.com/bladekp}.

Finally, I describe the system tests under the simulated rescue operation conditions, these tests take place during the 'Droniada' competition. The competition went well, but use of the internet of things and UAVs in rescue operations, is not good idea in my opinion, as I am writing at the very end of the paper.

\vskip 2cm
\noindent
\textbf{Keywords:}: IoT, UAV, RSSI, beacon, drone
\vskip 2cm
\noindent
(date and signature of graduate student)\hfill (date and signature of mentor)
