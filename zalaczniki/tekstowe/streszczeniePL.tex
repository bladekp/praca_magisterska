\newpage
\begin{center}
 {\large\bf  Streszczenie} \\
\vskip 1cm
Temat pracy:\\
\textit{\bf Wykorzystanie internetu rzeczy oraz bezzałogowych
statków powietrznych przy wspomaganiu prowadzenia
akcji ratunkowej}\\
\end{center}

Niniejsza praca, przedstawia proces tworzenia systemu wykorzystującego internet rzeczy i bezzałogowe statki powietrzne, przy wspomaganiu akcji ratowniczej. Zaczynam od wstępu teoretycznego, opisu technologii i sposobu w jaki chcę połączyć obie technologie, później przechodzę do swojej implementacji, opisu zastosowanych przeze mnie rozwiązań, problemów i ich rozwiązań, na koniec przedstawiam wyniki testów systemu i wnioski, jakie wyciągnąłem z tego projektu.

Temat jest trudny, trzeba połączyć dwie, dość nowe technologie, bezzałogowe statki powietrzne oraz internet rzeczy, tak żeby ze sobą współpracowały tworząc system, który z założenia ma pomagać ratownikom. Po drodze napotkałem wiele problemów, począwszy od bezpieczeństwa, przez autonomiczne sterowanie, krótki czas lotu UAV itd. Starałem się opisać, jak sobie z nimi radziłem.

W procesie tworzenia systemu powstało kilka programów, program na system Android, został zmodyfikowany program kontrolera lotu Pixhawk, oraz program stacji naziemnej w frameworku Django. Nie załączam tutaj kodu źródłowego, jest on natomiast dostępny w publicznych repozytoriach, na platformie Github \url{https://github.com/bladekp}. 

Na koniec, opisuję testy systemu w warunkach symulowanej akcji ratunkowej, które miały miejsce podczas konkursu "Droniada". Konkurs wypadł dobrze, natomiast samo zastosowanie internetu rzeczy i UAV przy prowadzeniu akcji ratunkowej, nie jest moim zdaniem dobrym pomysłem, o czym szerzej piszę na samym końcu pracy.

\vskip 2cm
\noindent
\textbf{Słowa kluczowe:} IoT, UAV, RSSI, beacon, dron
\vskip 2cm
\noindent
(data i podpis dyplomanta)\hfill (data i podpis opiekuna)
